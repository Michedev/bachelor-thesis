\chapter{Miglioramenti e conclusioni}
\section{Possibili miglioramenti per la raccolta di dati}
\begin{itemize}
	\item L'architettura \textit{client-server} e' necessaria per avere un'applicazione che riesca a gestire con fluidita' la ricerca all'interno di grandi edifici. In rifermento alla memoria del telefono poiche' nella maggior parte dei casi possiede un quantitativo di \textit{RAM} e memoria permanente molto limitati, ma anche rispetto al processore perche' la ricerca della posizione, assumendo di usare il \textit{KNN}, ha bisogno di confrontare gli attributi delle nuove onde magnetiche con tutti gli altri e va da se che piu' e' grande il dataset, piu' potenza di calcolo ci vorra' per ottenere una risposta in tempi umani.
	\item Durante la raccolta sarebbe opportuno applicare il \textit{filtro di Kalman} per ridurre il rumore causato dall'imprecisione dei sensori. E' stata usata in fase di test ma non nel codice che gira su dispositivi \textit{mobile}
	\item Per migliorare la precisione sarebbe opportuno appoggiarsi anche ad altri sensori presenti sullo \textit{smartphone}. Prendiamo come esempio il Wi-Fi, se il dispositivo e' connesso alla rete dell'edificio potrebbe sfruttare la potenza di segnale per avere una precisione maggiore; sfruttare l'accelerometro per stimare la velocita' del telefono e quindi standardizzare tutte le rilevazioni effettuate col magnetometro. Questa tecnica viene chiamata \textit{sensor fusion}\cite{shala2011indoor}
\end{itemize}
\section{Conclusioni}
In conclusione abbiamo una base di applicazione Android che, tramite l'uso di tecniche di \textit{machine learning} abbastanza semplici nella loro complessita' rispetto ad altre, ma che offrono comunque un buon grado di accuratezza anche se bisogna stare attenti a non cadere nella trappola dell'\textit{overfitting}. C'e' ancora da migliorare, come abbiamo visto nella sezione precedente, in tutti gli altri campi dell'applicazione, dall'interfaccia grafica alle risorse dedicate all'applicazione fino alle tecniche di raccolta dati.