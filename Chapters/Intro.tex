\chapter{Introduzione}
La seguente tesi \`e basata su un tirocinio esterno svolto con l'azienda $KeepUp$ in cui \`e stata sviluppata la base di un'applicazione Android col compito di localizzare all'interno degli edifici la posizione dello \textit{smartphone} sfruttando le distorsioni del campo magnetico terrestre\cite{6418880}.
\\\\
Un'applicazione del genere ha molti usi in luoghi chiusi aperti al pubblico: per esempio immaginiamoci di trovarci in un museo. In questo caso l'applicazione ufficiale del museo che supporta l'\textit{indoor positioning} ci localizza all'interno di una mappa $2D$ dell'edificio perci\`o riesce a capire se ci avviciniamo ad un'opera d'arte e quando avviene, si apre un \textit{pop-up} che ci fornisce informazioni aggiuntive su ci\`o che stiamo osservando. Oppure immaginiamoci di dover prendere l'aereo e di essere in ritardo dentro un'aeroporto che non conosciamo affatto. Un navigatore \textit{indoor} sarebbe molto utile a chi si trova in questa situazione perch\'e gli permetterebbe di raggiungere il proprio \textit{gate} in pochissimo tempo senza conoscere la piantina dell'aeroporto.
\\\\
Il primo capitolo \`e questo, una semplice introduzione al lavoro svolto.
\\\\
Durante il secondo capitolo, i fondamentali, approfondiremo il tipo di dato che dobbiamo trattare, le sue origini e la sua struttura per poi passare all'apprendimento automatico, usato per predire la posizione dell'utente all'interno dell'edificio elencandone i vari tipi e definendo alcuni termini gergali per concludere con la descrizione approfondita di alcuni classificatori molto conosciuti nel mondo dell'AI.
\\\\
Nel terzo capitolo parleremo della struttura del software \textit{Android} realizzato durante il tirocinio, dai linguaggi e librerie usate alla struttura del codice implementato.
\\\\
Nel quarto capitolo vedremo i risultati ottenuti dai dati estratti della nostra applicazione tramite vari grafici che mostrano i vari punti di forza e debolezza.
\\\\
Nel quinto capitolo trattiamo dei possibili miglioramenti futuri all'applicazione e delle conclusioni.