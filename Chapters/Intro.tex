\chapter{Introduzione}
La seguente tesi e' basata su un tirocinio esterno svolto con l'azienda $KeepUp$ in cui e' stata sviluppata la base di un'applicazione Android col compito di localizzare all'interno degli edifici la posizione dello \textit{smartphone} sfruttando le distorsioni del campo magnetico terrestre.
\\\\
Durante il primo capitolo, i fondamentali, approfondiremo il tipo di dato che dobbiamo trattare, le sue origini e la sua struttura per poi passare all'apprendimento automatico, usato per predire la posizione dell'utente all'interno dell'edificio elencandone i vari tipi e definendo alcuni termini gergali per concludere con la descrizione approfondita di alcuni classificatori molto conosciuti nel mondo dell'AI.
\\\\
Nel secondo capitolo parleremo della struttura del software realizzato durante il tirocinio, dal codice mobile Android a quello per la verifica dei risultati.
\\\\
Nel terzo capitolo vedremo i risultati ottenuti durante il tirocinio tramite vari grafici che mostreranno punti di forza e debolezza della nostra applicazione.
\\\\
Nel quarto capitolo discuteremo dei possibili miglioramenti futuri all'applicazione.