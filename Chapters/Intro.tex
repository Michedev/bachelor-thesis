\chapter{Introduzione}
La seguente tesi e' basata su un tirocinio esterno svolto con l'azienda $KeepUp$ in cui e' stata sviluppata la base di un'applicazione Android col compito di localizzare all'interno degli edifici la posizione dello \textit{smartphone} sfruttando le distorsioni del campo magnetico terrestre.
\\\\
Un'applicazione del genere ha molti usi in luoghi chiusi aperti al pubblico: per esempio immaginiamoci di trovarci in un museo. In questo caso l'applicazione ufficiale del museo che supporta l'\textit{indoor positioning} ci localizza all'interno di una mappa $2D$ dell'edificio. L'applicazione capisce quando ci avviciniamo ad un'opera d'arte e quando avviene, si apre un \textit{pop-up} che ci fornisce informazioni aggiuntive su cio' che stiamo osservando. Oppure immaginiamoci di dover prendere l'aereo e di essere in ritardo in un aeroporto all'estero che non conosciamo affatto. Un navigatore \textit{inddor} potrebbe far molto comodo a chi si trova in questa situazione perche' gli permetterebbe di raggiungere il proprio \textit{gate} in pochissimo tempo senza conoscere la piantina dell'aeroporto.
\\\\
Durante il primo capitolo, i fondamentali, approfondiremo il tipo di dato che dobbiamo trattare, le sue origini e la sua struttura per poi passare all'apprendimento automatico, usato per predire la posizione dell'utente all'interno dell'edificio elencandone i vari tipi e definendo alcuni termini gergali per concludere con la descrizione approfondita di alcuni classificatori molto conosciuti nel mondo dell'AI.
\\\\
Nel secondo capitolo parleremo della struttura del software realizzato durante il tirocinio, dal codice mobile Android a quello per la verifica dei risultati.
\\\\
Nel terzo capitolo vedremo i risultati ottenuti durante il tirocinio tramite vari grafici che mostreranno punti di forza e debolezza della nostra applicazione.
\\\\
Nel quarto capitolo discuteremo dei possibili miglioramenti futuri all'applicazione.